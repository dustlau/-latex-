%!TEX program = xelatex
\documentclass[a4paper]{book}
\usepackage{ctex}
\usepackage{xeCJK}
\usepackage{fancyhdr}
\usepackage{enumitem}  %列表间距


\begin{document}
%========论文封面==========
% \frontmatter
% \secretlevel{绝密} \secretyear{2100}

% \ctitle{融合内容信息的单类协同过滤研究}
% \cauthor{徐留成}

% \cmajor{计算机科学与技术}

% \cdepartment{计算机与软件学院}
% \studentID{2012080173}
 
% \csupervisor{潘微科}
% \professionaltitle{讲师 }
\begin{titlepage}

\end{titlepage}
%=========原创性声明与授权说明=========
%========摘要中文==========
\frontmatter
%  \chapter {摘\quad 要}\chaptermark{摘\quad 要}  %%属于哪个包
\chapter[中文摘要]{摘\quad 要}
%方括号中是在目录中显示的内容,相当于前面使用的\addcontentsline中文摘要内容
\pagestyle{fancy}
\pagenumbering{Roman} %大写罗马字体
\lhead{}
\chead{\bfseries 新巴塞尔协议风险管理理念与我国风险管理体系的构建}
\rhead{}
\lfoot{}
\cfoot{\thepage}
\rfoot{}
% \begin{abstract}
% % \keywords{\TeX, \LaTeX, CJK, 模板, 论文}
% \end{abstract}

% \renewcommand{\headrulewidth}{0.4pt}
% \renewcommand{\footrulewidth}{0.4pt}
%==========================================
 % \maketitle
%==========================================


%==========================================
%=========摘要英文=========
% \pagestyle{fancy}
% \pagenumbering{Roman} %大写罗马字体
% \lhead{}
% \chead{\bfseries The Idea of Global Risk Management from the New Basle Capital Accord and the ERM,s Constructing of in China}
% \rhead{}
% \lfoot{}
% \cfoot{\thepage}
% \rfoot{}
% \begin{abstract}
% \end{abstract}
%========目录==========
%=========引言或绪论=========
% \author{Mary\thanks{E-mail:*****@***.com}}
% pt 点阵宽度,1/72.27in
% bp 点阵宽度,1/72in
% in 英寸
% cm 厘米
% mm 毫米
% em 当前字号下大写字母 M 的宽度,常用于水平距离的设定
% ex 当前字号下小写字母 x 的高度,常用于垂直距离的设定
%=========正文=========
%%宋体的 BoldFont 配置为黑体,而 ItalicFont 配置为楷体
% %%如果不是在导言区全局修改,而想要局部地改变某个段落的行距,需要用 \selectfont 命令使 \linespread 命令的改动立即生效
%%===============有序============================
% \begin{enumerate}
% \item An item.
% \begin{enumerate}
% \item A nested item.
% \item[*] A starred item.
% \item Another item. \label{itref}
% \end{enumerate}
% \item Go back to upper level.
% \item Reference(\ref{itref}).
% \end{enumerate}
%%=============对齐方式=======================% \centering \raggedright \raggedleft
%%=============图表=======================

%====================================
%====================================
%====================================
%====================================
%====================================
%====================================
%====================================
%====================================
%====================================
%====================================
%====================================
%====================================
%====================================
%====================================
%==========结论========
%==========参考文献========
\backmatter     % 开始正文之后的部分
\begin{thebibliography}{99}\addcontentsline{toc}{chapter}{参考文献}
参考文献内容
 \end{thebibliography}
%=========附录========
%=========致谢=========
\chapter[致谢]{致\quad 谢}
%========攻读硕士期间研究成果==========
\end{document}
%==================
%==================
%==================
%==================
%==================
%==================
%==================